\documentclass[journal,12pt,twocolumn]{IEEEtran}

\usepackage{setspace}
\usepackage{gensymb}

\singlespacing


\usepackage[cmex10]{amsmath}

\usepackage{amsthm}

\usepackage{mathrsfs}
\usepackage{txfonts}
\usepackage{stfloats}
\usepackage{bm}
\usepackage{cite}
\usepackage{cases}
\usepackage{subfig}

\usepackage{longtable}
\usepackage{multirow}

\usepackage{enumitem}
\usepackage{mathtools}
\usepackage{steinmetz}
\usepackage{tikz}
\usepackage{circuitikz}
\usepackage{verbatim}
\usepackage{tfrupee}
\usepackage[breaklinks=true]{hyperref}
\usepackage{graphicx}
\usepackage{tkz-euclide}
\usepackage{float}

\usetikzlibrary{calc,math}
\usepackage{listings}
    \usepackage{color}                                            %%
    \usepackage{array}                                            %%
    \usepackage{longtable}                                        %%
    \usepackage{calc}                                             %%
    \usepackage{multirow}                                         %%
    \usepackage{hhline}                                           %%
    \usepackage{ifthen}                                           %%
    \usepackage{lscape}     
\usepackage{multicol}
\usepackage{chngcntr}

\DeclareMathOperator*{\Res}{Res}

\renewcommand\thesection{\arabic{section}}
\renewcommand\thesubsection{\thesection.\arabic{subsection}}
\renewcommand\thesubsubsection{\thesubsection.\arabic{subsubsection}}

\renewcommand\thesectiondis{\arabic{section}}
\renewcommand\thesubsectiondis{\thesectiondis.\arabic{subsection}}
\renewcommand\thesubsubsectiondis{\thesubsectiondis.\arabic{subsubsection}}


\hyphenation{op-tical net-works semi-conduc-tor}
\def\inputGnumericTable{}                                 %%

\lstset{
%language=C,
frame=single, 
breaklines=true,
columns=fullflexible
}
\makeatletter
\setlength{\@fptop}{0pt}
\makeatother
\begin{document}
\newtheorem{theorem}{Theorem}[section]
\newtheorem{problem}{Problem}
\newtheorem{proposition}{Proposition}[section]
\newtheorem{lemma}{Lemma}[section]
\newtheorem{corollary}[theorem]{Corollary}
\newtheorem{example}{Example}[section]
\newtheorem{definition}[problem]{Definition}

\newcommand{\BEQA}{\begin{eqnarray}}
\newcommand{\EEQA}{\end{eqnarray}}
\newcommand{\define}{\stackrel{\triangle}{=}}
\bibliographystyle{IEEEtran}
\providecommand{\mbf}{\mathbf}
\providecommand{\pr}[1]{\ensuremath{\Pr\left(#1\right)}}
\providecommand{\qfunc}[1]{\ensuremath{Q\left(#1\right)}}
\providecommand{\sbrak}[1]{\ensuremath{{}\left[#1\right]}}
\providecommand{\lsbrak}[1]{\ensuremath{{}\left[#1\right.}}
\providecommand{\rsbrak}[1]{\ensuremath{{}\left.#1\right]}}
\providecommand{\brak}[1]{\ensuremath{\left(#1\right)}}
\providecommand{\lbrak}[1]{\ensuremath{\left(#1\right.}}
\providecommand{\rbrak}[1]{\ensuremath{\left.#1\right)}}
\providecommand{\cbrak}[1]{\ensuremath{\left\{#1\right\}}}
\providecommand{\lcbrak}[1]{\ensuremath{\left\{#1\right.}}
\providecommand{\rcbrak}[1]{\ensuremath{\left.#1\right\}}}
\theoremstyle{remark}
\newtheorem{rem}{Remark}
\newcommand{\sgn}{\mathop{\mathrm{sgn}}}
\providecommand{\abs}[1]{\vert#1\vert}
\providecommand{\res}[1]{\Res\displaylimits_{#1}} 
\providecommand{\norm}[1]{\lVert#1\rVert}
%\providecommand{\norm}[1]{\lVert#1\rVert}
\providecommand{\mtx}[1]{\mathbf{#1}}
\providecommand{\mean}[1]{E[ #1 ]}
\providecommand{\fourier}{\overset{\mathcal{F}}{ \rightleftharpoons}}
%\providecommand{\hilbert}{\overset{\mathcal{H}}{ \rightleftharpoons}}
\providecommand{\system}{\overset{\mathcal{H}}{ \longleftrightarrow}}
	%\newcommand{\solution}[2]{\textbf{Solution:}{#1}}
\newcommand{\solution}{\noindent \textbf{Solution: }}
\newcommand{\cosec}{\,\text{cosec}\,}
\providecommand{\dec}[2]{\ensuremath{\overset{#1}{\underset{#2}{\gtrless}}}}
\newcommand{\myvec}[1]{\ensuremath{\begin{pmatrix}#1\end{pmatrix}}}
\newcommand{\mydet}[1]{\ensuremath{\begin{vmatrix}#1\end{vmatrix}}}
\numberwithin{equation}{subsection}
\makeatletter
\@addtoreset{figure}{problem}
\makeatother
\let\StandardTheFigure\thefigure
\let\vec\mathbf
\renewcommand{\thefigure}{\theproblem}
\def\putbox#1#2#3{\makebox[0in][l]{\makebox[#1][l]{}\raisebox{\baselineskip}[0in][0in]{\raisebox{#2}[0in][0in]{#3}}}}
     \def\rightbox#1{\makebox[0in][r]{#1}}
     \def\centbox#1{\makebox[0in]{#1}}
     \def\topbox#1{\raisebox{-\baselineskip}[0in][0in]{#1}}
     \def\midbox#1{\raisebox{-0.5\baselineskip}[0in][0in]{#1}}
\vspace{3cm}
\title{ASSIGNMENT 5}
\author{Vishwanath Hurakadli \\ AI20BTECH11023}
\maketitle
\newpage
\bigskip
\renewcommand{\thefigure}{\theenumi}
\renewcommand{\thetable}{\theenumi}
Download all python codes from 
\begin{lstlisting}
https://github.com/vishwahurakadli/EE3900/blob/main/Assignment_5/EE3900_Assignment_5.ipynb
\end{lstlisting}
%
and latex-tikz codes from 
%
\begin{lstlisting}
https://github.com/vishwahurakadli/EE3900/blob/main/Assignment_5/EE3900_Assignment_5.tex
\end{lstlisting}
\section{Quadratic forms 2.26}
Find the coordinates of the foci, the vertices,the length of major axis,minor axis the
eccentricity and the latus rectum of the ellipse of the ellipse
\begin{align}
  \vec{x}^T\myvec{\frac{1}{25}&0\\0&\frac{1}{9}}\vec{x}=9   
\end{align}
%
\section{SOLUTION}
%
Given ellipse is
\begin{align}
   \vec{x}^T\myvec{\frac{1}{25}&0\\0&\frac{1}{9}}\vec{x}&=1\\
   \text{can also be written as}\\
   \vec{x}^T\myvec{9&0\\0&25}\vec{x}&=225
\end{align}
%
On comparing it with standard form we have,
\begin{align}
    \vec{V}=\myvec{9&0\\0&25},\vec{u}=0,f=-225\\
    \implies\vec{u}^T\vec{V}^{-1}\vec{u}-f = 225\\
    \implies\vec{c} = -\vec{V}^{-1}\vec{u} = \myvec{0\\0}\\
\end{align}
The eigen vector decomposition of 
\begin{align}
    \vec{V} = \myvec{9&0\\0&25}
\end{align}
is given by
\begin{align}
    \vec{D} &= \myvec{9&0\\0&25} \implies \lambda_1 = 9,\lambda_2 = 25\\
    \vec{P} &= \myvec{1&0\\0&1}\implies \vec{p}_1 = \myvec{1\\0}, \vec{p}_2 = \myvec{0\\1}
\end{align}
Since
\begin{align}
   \lambda_1 < \lambda_2
\end{align}
Eccentricity of the ellipse is,
\begin{align}
   e &= \sqrt{1-\frac{\lambda_1}{\lambda_2}}
   = \frac{4}{5}
\end{align} 
Semi major and minor axes of ellipse are,
\begin{align}
    a = \sqrt{\frac{\vec{u}^{\top}\vec{V}^{-1}\vec{u}-f}{\lambda_1}} = 5\\
    b = \sqrt{\frac{\vec{u}^{\top}\vec{V}^{-1}\vec{u}-f}{\lambda_2}} = 3
\end{align}
Length of major axis, minor axis and latus rectum is given by
\begin{align}
M = 2a = 10\\
m= 2b = 6\\
L =2\frac{{\sqrt{\frac{\vec{u}^{\top}\vec{V}^{-1}\vec{u}-f}{\lambda_2}}}^2}{\sqrt{\frac{\vec{u}^{\top}\vec{V}^{-1}\vec{u}-f}{\lambda_1}}}
&= \frac{2b^2}{a}= \frac{18}{5}
\end{align}
The co-ordinates of vertices are,
\begin{align}
   \pm\myvec{5 \\ 0} 
\end{align}
The co-ordinates of foci are given by,
\begin{align}
  \vec{F}  = \frac{ce^2\vec{n}-\vec{u}}{\lambda_2}
\end{align}
Where,
\begin{align}
    \vec{n} &= \sqrt{\lambda_2}\vec{p}_1
\end{align}
\begin{multline}
     c = \frac{e\vec{u}^{\top}\vec{n} \pm \sqrt{e^2\brak{\vec{u}^{\top}\vec{n}}^2-\lambda_1\brak{e^2-1}\brak{\norm{\vec{u}}^2 - \lambda_1 f}}}{\lambda_1e\brak{e^2-1}} 
\end{multline}
Substituting we have,
\begin{align}
    \vec{n} &= \myvec{5\\0}\\ c &= \pm\frac{125}{4}\\
    \vec{F} &= \pm\myvec{4 \\ 0}.
\end{align}

\begin{figure}[h!]
\centering
\includegraphics[width=\columnwidth]{EE3900_Graph.png}
\caption{Plot of ellipse}
\label{fig:ellipse}
\end{figure}
\end{document}
