\documentclass[journal,12pt,twocolumn]{IEEEtran}

\usepackage{setspace}
\usepackage{gensymb}

\singlespacing


\usepackage[cmex10]{amsmath}

\usepackage{amsthm}

\usepackage{mathrsfs}
\usepackage{txfonts}
\usepackage{stfloats}
\usepackage{bm}
\usepackage{cite}
\usepackage{cases}
\usepackage{subfig}

\usepackage{longtable}
\usepackage{multirow}

\usepackage{enumitem}
\usepackage{mathtools}
\usepackage{steinmetz}
\usepackage{tikz}
\usepackage{circuitikz}
\usepackage{verbatim}
\usepackage{tfrupee}
\usepackage[breaklinks=true]{hyperref}
\usepackage{graphicx}
\usepackage{tkz-euclide}
\usepackage{float}

\usetikzlibrary{calc,math}
\usepackage{listings}
    \usepackage{color}                                            %%
    \usepackage{array}                                            %%
    \usepackage{longtable}                                        %%
    \usepackage{calc}                                             %%
    \usepackage{multirow}                                         %%
    \usepackage{hhline}                                           %%
    \usepackage{ifthen}                                           %%
    \usepackage{lscape}     
\usepackage{multicol}
\usepackage{chngcntr}

\DeclareMathOperator*{\Res}{Res}

\renewcommand\thesection{\arabic{section}}
\renewcommand\thesubsection{\thesection.\arabic{subsection}}
\renewcommand\thesubsubsection{\thesubsection.\arabic{subsubsection}}

\renewcommand\thesectiondis{\arabic{section}}
\renewcommand\thesubsectiondis{\thesectiondis.\arabic{subsection}}
\renewcommand\thesubsubsectiondis{\thesubsectiondis.\arabic{subsubsection}}


\hyphenation{op-tical net-works semi-conduc-tor}
\def\inputGnumericTable{}                                 %%

\lstset{
%language=C,
frame=single, 
breaklines=true,
columns=fullflexible
}
\begin{document}
\newtheorem{theorem}{Theorem}[section]
\newtheorem{problem}{Problem}
\newtheorem{proposition}{Proposition}[section]
\newtheorem{lemma}{Lemma}[section]
\newtheorem{corollary}[theorem]{Corollary}
\newtheorem{example}{Example}[section]
\newtheorem{definition}[problem]{Definition}

\newcommand{\BEQA}{\begin{eqnarray}}
\newcommand{\EEQA}{\end{eqnarray}}
\newcommand{\define}{\stackrel{\triangle}{=}}
\bibliographystyle{IEEEtran}
\providecommand{\mbf}{\mathbf}
\providecommand{\pr}[1]{\ensuremath{\Pr\left(#1\right)}}
\providecommand{\qfunc}[1]{\ensuremath{Q\left(#1\right)}}
\providecommand{\sbrak}[1]{\ensuremath{{}\left[#1\right]}}
\providecommand{\lsbrak}[1]{\ensuremath{{}\left[#1\right.}}
\providecommand{\rsbrak}[1]{\ensuremath{{}\left.#1\right]}}
\providecommand{\brak}[1]{\ensuremath{\left(#1\right)}}
\providecommand{\lbrak}[1]{\ensuremath{\left(#1\right.}}
\providecommand{\rbrak}[1]{\ensuremath{\left.#1\right)}}
\providecommand{\cbrak}[1]{\ensuremath{\left\{#1\right\}}}
\providecommand{\lcbrak}[1]{\ensuremath{\left\{#1\right.}}
\providecommand{\rcbrak}[1]{\ensuremath{\left.#1\right\}}}
\theoremstyle{remark}
\newtheorem{rem}{Remark}
\newcommand{\sgn}{\mathop{\mathrm{sgn}}}
\providecommand{\abs}[1]{\vert#1\vert}
\providecommand{\res}[1]{\Res\displaylimits_{#1}} 
\providecommand{\norm}[1]{\lVert#1\rVert}
%\providecommand{\norm}[1]{\lVert#1\rVert}
\providecommand{\mtx}[1]{\mathbf{#1}}
\providecommand{\mean}[1]{E[ #1 ]}
\providecommand{\fourier}{\overset{\mathcal{F}}{ \rightleftharpoons}}
%\providecommand{\hilbert}{\overset{\mathcal{H}}{ \rightleftharpoons}}
\providecommand{\system}{\overset{\mathcal{H}}{ \longleftrightarrow}}
	%\newcommand{\solution}[2]{\textbf{Solution:}{#1}}
\newcommand{\solution}{\noindent \textbf{Solution: }}
\newcommand{\cosec}{\,\text{cosec}\,}
\providecommand{\dec}[2]{\ensuremath{\overset{#1}{\underset{#2}{\gtrless}}}}
\newcommand{\myvec}[1]{\ensuremath{\begin{pmatrix}#1\end{pmatrix}}}
\newcommand{\mydet}[1]{\ensuremath{\begin{vmatrix}#1\end{vmatrix}}}
\numberwithin{equation}{subsection}
\makeatletter
\makeatother
\let\StandardTheFigure\thefigure
\let\vec\mathbf
\renewcommand{\thefigure}{\theproblem}
\def\putbox#1#2#3{\makebox[0in][l]{\makebox[#1][l]{}\raisebox{\baselineskip}[0in][0in]{\raisebox{#2}[0in][0in]{#3}}}}
     \def\rightbox#1{\makebox[0in][r]{#1}}
     \def\centbox#1{\makebox[0in]{#1}}
     \def\topbox#1{\raisebox{-\baselineskip}[0in][0in]{#1}}
     \def\midbox#1{\raisebox{-0.5\baselineskip}[0in][0in]{#1}}
\vspace{3cm}
\title{ASSIGNMENT 1}
\author{Vishwanath Hurakadli\\ AI20BTECH11023}
\maketitle
\newpage
\bigskip
\renewcommand{\thefigure}{\theenumi}
\renewcommand{\thetable}{\theenumi}
Download all latex-tikz codes from 
\begin{lstlisting}
https://github.com/vishwahurakadli/EE3900/blob/main/Assignment_1/Assignment_1.tex
\end{lstlisting}
%
\section{Problem}
(Vectors-2.20) If
\begin{align}
\vec{P} = 3\vec{a} - 2\vec{b}\\
\vec{Q} = \vec{a} +\vec{b}
\end{align}
 find $\vec{R}$ which divides PQ in the ratio 2 : 1
\begin{enumerate}
    \item internally \label{option A}
    \item externally \label{option B}
\end{enumerate}
\section{Solution}
Vector $\vec{P}$ and $\vec{Q}$ can be represented using $\vec{a}$ and $\vec{b}$ as
\begin{align}
\vec{P} &= \myvec{3&-2}\myvec{\vec{a}\\\vec{b}} = 3\vec{a} - 2\vec{b}\\
\vec{Q} &= \myvec{1&1}\myvec{\vec{a}\\\vec{b}} = \vec{a} +\vec{b}
\end{align}
\begin{enumerate}
\item 
section formula for internal division for ratio m : n is given by 
\begin{align}
\vec{I} &= \myvec{\frac{m}{m+n} & \frac{n}{m+n}}\myvec{\vec{P}\\\vec{Q}}\\
&=  \frac{m\vec{P} + n\vec{Q}}{m + n}
\end{align}
so for ratio 2 : 1 R will be given by
\begin{align}
\vec{R} &= \myvec{\frac{2}{2+1} & \frac{1}{2+1}}\myvec{\vec{P}\\\vec{Q}}\\
&= \frac{2}{3}\vec{P} + \frac{1}{3}\vec{Q}\\
&= \myvec{2 & -\frac{4}{3}}\myvec{\vec{a}\\\vec{b}} + 
\myvec{\frac{1}{3} & \frac{1}{3}}\myvec{\vec{a}\\\vec{b}}\\
&= \myvec{\frac{7}{3}&-1}\myvec{\vec{a}\\\vec{b}}\\
\vec{R} &= \frac{7}{3}\vec{a} - \vec{b}
\end{align}
$\vec{R}$ will divide PQ internally\\
\item similarly section formula for external division for ration m : n is given by
\begin{align}
\vec{E} &= \myvec{\frac{m}{m-n} & \frac{n}{m-n}}\myvec{\vec{P}\\\vec{Q}}\\
&= \frac{m\vec{P} - n\vec{Q}}{m - n}
\end{align}
so for ratio 2 : 1 R will be given by
\begin{align}
\vec{R} &= \myvec{\frac{2}{2-1} & -\frac{1}{2-1}}\myvec{\vec{P}\\\vec{Q}}\\
&= 2\vec{P} - \vec{Q}\\
&= \myvec{6 & -4}\myvec{\vec{a}\\\vec{b}} + 
\myvec{-1 & -1}\myvec{\vec{a}\\\vec{b}}\\
&= \myvec{5&-5}\myvec{\vec{a}\\\vec{b}}\\
\vec{R} &= 5\vec{a} - 5\vec{b}
\end{align}
$\vec{R}$ will divide PQ externally
\end{enumerate}
\end{document}
